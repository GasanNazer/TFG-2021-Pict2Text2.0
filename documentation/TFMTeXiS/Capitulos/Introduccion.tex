% !TeX encoding = ISO-8859-1

\chapter{Introduction}
\label{cap:introduction}

This chapter will display the motivation behind this project and its objectives. In the end, we present the document structure with the different sections within it and a brief description of them.

\section{Motivation}
Communication is a pillar in interpersonal relationships, a fundamental need in our society. However, for some people communication requires a lot of effort. Their differences create an uncrossable barrier and almost impossible human connection using the traditional way of communication. To remove this barrier an alternative approach should be used, for example the use of alternative ways of communication as pictograms. These graphic images, representing an object or a concept, have helped people with special needs to communicate. However, the majority of the population does not understand pictograms.

In order to include pictograms into the communication between people with disabilities and those without, we can use modern technology and create a software to be a mediator between the two sides. 

Currently, multiple tools allow transforming natural language into pictograms, one of which is the first version of this project -  Pict2Text, but there is no such tool that does the inverse - from an image of pictograms to text. Pict2Text translates pictograms into natural language, but to create the text with pictograms, the pictograms must be selected manually searching for every single one in a search engine by entering the words associated with the pictograms. This is not good enough as people with disabilities who need the pictograms, cannot think about the words that compose the sentence and search for them in the search engine. This problem can be solved, giving those people the option to upload or take a picture of a sentence constructed with pictograms. By creating a new version of Pict2Text, we aim to reduce the exclusion of those people with special needs from the society and break the communication barrier, providing the aforementioned functionality. Another objective of ours is the improvement of the translation since currently only simple sentences can be translated.

The beneficiaries of this software are people who don't understand pictograms but want to communicate with people who are using them. This project will help people understand each other better, creating a more equal society, independently of the ability to use the natural language in our communication. 

\section{Goals}
The main goal of our project is to improve Pict2Text. To do that, we have two objectives - to build an application that can recognize pictograms on uploaded or taken at the moment pictures and improve the translation that Pict2Text provides given a set of pictograms.

We use services-oriented application architecture, constructing web services, and microservices, to increase maintainability and scalability. For the same reasons, we also follow industry standards during the whole process.

Last but not least, we want to consolidate and expand the knowledge acquired during the Software Engineering Degree.

\section{Document structure}
The document is structured as follows.
Chapter 2, State of the Art, covers the state of the first version of Pict2Text, the functionalities we will include to increase its usefulness, and the general idea of the machine learning algorithms and models we will be using.
Chapter 3, Software development methodology, describes the methodology we have chosen to use, it's characteristics, and things specific to our application of it. 







