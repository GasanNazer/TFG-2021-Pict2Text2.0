% !TeX encoding = ISO-8859-1
% +--------------------------------------------------------------------+
% | Copyright Page
% +--------------------------------------------------------------------+

\chapter*{Resumen}
Hoy en d�a, la comunicaci�n es una necesidad b�sica en nuestra sociedad. Sin embargo, algunas personas no pueden utilizar los m�todos t�picos de comunicaci�n por razones que no dependen de ellos. Una forma de que esas personas se comuniquen es utilizando pictogramas. Sin embargo, para las personas sin formaci�n espec�fica, entender las frases formadas por esos pictogramas no es f�cil, y a veces puede llegar a ser imposible. Por eso, las herramientas que traducen al lenguaje natural las frases escritas con pictogramas, son esenciales.

Pict2Text 1.0 es la �nica herramienta existente que traduce mensajes escritos con pictogramas al lenguaje natural (espa�ol). Por desgracia, todav�a tiene que ser mejorada. Uno de los mayores defectos de la herramienta es el hecho de que el mensaje con pictogramas tiene que crearse manualmente buscando cada pictograma en el buscador que proporciona la aplicaci�n. En este estado actual, las personas que m�s necesitan la herramienta no pueden utilizarla ya que no son capaces de escribir las palabras que componen el mensaje para seleccionar los pictogramas. Por eso, en este proyecto final, hemos centrado nuestros esfuerzos en mejorar esa funci�n dando a las personas la opci�n de subir una imagen de una frase escrita con pictogramas en lugar de crear el mensaje de la forma mencionada. Para ello, hemos implementado y probado dos modelos de Machine Learning: uno para detectar los pictogramas de una imagen (YOLO) y otro para identificar la palabra asociada a cada uno de los pictogramas de una imagen (algoritmo One-shot learning). Hemos entrenado y probado varias versiones de cada uno de ellos. Hemos conseguido reconocer m�s de un pictograma en una imagen ampliada. El algoritmo de clasificaci�n es capaz de reconocer correctamente alrededor del 70\% de los pictogramas. Aunque queda trabajo por hacer, los resultados que hemos obtenido son lo suficientemente alentadores como para creer que si aumentamos el conjunto de entrenamiento, la precisi�n de la predicci�n ser� m�s que satisfactoria.
 
\clearpage  



\section*{Palabras clave}
   
\begin{itemize}
    \item Pictogramas.
    \item ARAASAC.
    \item Machine Learning.
    \item Visi�n artificial.
    \item One-shot learning.
    \item YOLO.
\end{itemize}

   


