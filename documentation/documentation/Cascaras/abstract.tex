% !TeX encoding = ISO-8859-1

\chapter*{Abstract}

Nowadays, communication is a basic need in our society. However, some people cannot use the typical methods of communication for reasons that don't depend on them. A way of communication for those people is using pictograms. However, for people without specific training, understanding sentences formed by those pictograms is not easy, if not impossible. That's why tools that translate sentences written with pictograms into natural language are essential.

Pict2Text 1.0 is the only existing tool that translates messages written with pictograms to natural language (Spanish). Unfortunately, it still has to be improved. One of the biggest flaws of the tool is the fact that the message with pictograms has to be created manually by looking for each pictogram in the search engine provided by the application. At this current state, the people who most need the tool can't use it because they would not be able to type the words to compound the message to select the pictograms. For that reason, in this final project, we have focused our efforts on improving that feature by giving people the option to upload a picture of a sentence written with pictograms instead of creating the message in the above-mentioned way. To do that, we have implemented and tested two Machine Learning models: one to detect the pictograms in a picture (YOLO) and the other to identify the word associated with each one of the pictograms in an image (One-shot learning algorithm). We trained and tested several versions of each of them. We have managed to recognize more than one pictogram in a zoomed-in picture. The classification algorithm is able to recognize around 70\% of the pictograms correctly. Even though there is work left, the results we obtained are encouraging enough to believe that if we increase the training set, the prediction accuracy will be more than satisfactory.

\clearpage  

\section*{Keywords}

\begin{itemize}
    \item Pictogram.
    \item ARAASAC.
    \item Machine Learning.
    \item Computer vision.
    \item One-shot learning.
    \item YOLO.
\end{itemize}



   


   
